\documentclass{article}

\usepackage{tabularx}
\usepackage{booktabs}
\usepackage{esvect}
\usepackage{mathtools}
\usepackage{bm}
\usepackage{amsmath, mathtools}
\usepackage{amssymb}



\title{Problem Statement and Goals\\\progname}

\author{Fasil Cheema}

%\date{}

%%% Comments

\usepackage{color}

\newif\ifcomments\commentstrue %displays comments
%\newif\ifcomments\commentsfalse %so that comments do not display

\ifcomments
\newcommand{\authornote}[3]{\textcolor{#1}{[#3 ---#2]}}
\newcommand{\todo}[1]{\textcolor{red}{[TODO: #1]}}
\else
\newcommand{\authornote}[3]{}
\newcommand{\todo}[1]{}
\fi

\newcommand{\wss}[1]{\authornote{blue}{SS}{#1}} 
\newcommand{\plt}[1]{\authornote{magenta}{TPLT}{#1}} %For explanation of the template
\newcommand{\an}[1]{\authornote{cyan}{Author}{#1}}

%%% Common Parts

\newcommand{\progname}{ProgName} % PUT YOUR PROGRAM NAME HERE
\newcommand{\authname}{Team \#, Team Name
\\ Student 1 name
\\ Student 2 name
\\ Student 3 name
\\ Student 4 name} % AUTHOR NAMES                  

\usepackage{hyperref}
    \hypersetup{colorlinks=true, linkcolor=blue, citecolor=blue, filecolor=blue,
                urlcolor=blue, unicode=false}
    \urlstyle{same}
                                


\begin{document}

\maketitle

\begin{table}[hp]
\caption{Revision History} \label{TblRevisionHistory}
\begin{tabularx}{\textwidth}{llX}
\toprule
\textbf{Date} & \textbf{Developer(s)} & \textbf{Change}\\
\midrule
Jan 19, 2024 & Fasil Cheema & Initial Upload\\
April 15, 2024 & Fasil Cheema & Revised Version\\
\bottomrule
\end{tabularx}
\end{table}

\section{Problem Statement}

The optimization of functions is a critical problem that is prevalent in many domains; such as engineering, operations research, physical analysis of systems, business analysis, and machine learning to name a few. To solve these problems is therefore a critical task and having a family of solvers that can properly find solutions successfully, quickly, and efficiently is vital. Therefore, to address this we seek to develop a library of functions to solve a multitude of optimization problems. For the scope of this project we choose to focus on two families of algorithms, the first of which is Conjugate Gradient methods; which includes methods such as the Fletcher-Reeves method. The other family of algorithms is the Quasi-Newton family which includes the Broyden-Fletcher-Goldfarb-Shanno (BFGS) method and the David-Fletcher-Powell (DFP) algorithm. We will restrict the library to work on functions of the quadratic form. That is $f(\vv{x}) = \vv{x}^{T}\mathbf{A}\vv{x} + \vv{b}\vv{x} + c$, where $\mathbf{A} \in \mathbb{R}^{d \times d}$,$\vv{x} \in \mathbb{R}^{d \times 1}$,$\vv{b} \in \mathbb{R}^{1 \times d}$, and $c \in \mathbb{R}$.
\\

\subsection{Problem}

\subsection{Inputs and Outputs}
The input to all the functions in the library of solvers would be a function $f$ which should be given in a certain appropriate parametric form, initial values and certain parametric choices left to the user that alter the solver's performance during the search (such as step size, or other parameters relevant to a specific algorithm).  So our problem is then given a function that satisfies certain constraints we should be able to use our library to output a solution which can either be a minimum or maximum (depending on the problem) to said problem, if possible.
\\

\subsection{Stakeholders}
Many domains apply optimization algorithms to obtain satisfactory solutions to their problems. Anyone who wishes to use this library to solve their domain-specific task is a valid stakeholder which can include anyone in scientific computing, engineering, medicine, or machine learning. However, the stakeholders would likely be restricted to the class of technically competent individuals with some scientific background in optimization. 
\subsection{Environment}
 The software should be reliable and be able to be run on multiple OS ranging from MacOS, Linux, UNIX, Windows and a multitude of devices (ranging from laptops to advanced clusters). Interpretability is also stressed so that future practitioners that seek to implement this library in their work can understand the functions. 

\section{Goals}
The major goal of the project is given a function (in an appropriate format), a set of parameters for the relevant algorithms, initial value (if relevant) we wish to return a solution to the minimization/maximization problem. If possible we would like to create a more expansive library (time permitting) however if not possible we may choose to restrict the scope of our project to one family of solvers.
\end{document}