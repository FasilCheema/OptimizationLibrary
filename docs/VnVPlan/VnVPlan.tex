\documentclass[12pt, titlepage]{article}

\usepackage{booktabs}
\usepackage{tabularx}
\usepackage{hyperref}
\hypersetup{
    colorlinks,
    citecolor=blue,
    filecolor=black,
    linkcolor=red,
    urlcolor=blue
}
\usepackage[round]{natbib}

\input{../Comments}
\input{../Common}

\begin{document}

\title{Optimization Library: System Verification and Validation Plan} 

\author{Fasil Cheema}
\date{\today}
	
\maketitle

\pagenumbering{roman}

\section*{Revision History}

\begin{tabularx}{\textwidth}{p{3cm}p{2cm}X}
\toprule {\bf Date} & {\bf Version} & {\bf Notes}\\
\midrule
Feb 19, 2023 & 1.0 & Initial Upload\\
\bottomrule
\end{tabularx}

~\\
\wss{The intention of the VnV plan is to increase confidence in the software.
However, this does not mean listing every verification and validation technique
that has ever been devised.  The VnV plan should also be a \textbf{feasible}
plan. Execution of the plan should be possible with the time and team available.
If the full plan cannot be completed during the time available, it can either be
modified to ``fake it'', or a better solution is to add a section describing
what work has been completed and what work is still planned for the future.}

\wss{The VnV plan is typically started after the requirements stage, but before
the design stage.  This means that the sections related to unit testing cannot
initially be completed.  The sections will be filled in after the design stage
is complete.  the final version of the VnV plan should have all sections filled
in.}

\newpage

\tableofcontents

\listoftables
\wss{Remove this section if it isn't needed}

\listoffigures
\wss{Remove this section if it isn't needed}

\newpage

\section{Symbols, Abbreviations, and Acronyms}

\renewcommand{\arraystretch}{1.2}
\begin{tabular}{l l} 
  \toprule		
  \textbf{symbol} & \textbf{description}\\
  \midrule 
  T & Test\\
  VnV & Verification and Validation\\
  FR & Functional Requirement\\
  NFR & Non-Functional Requirement\\
  MG & Module Guide\\
  MIS & Module Interface Specification\\
  \bottomrule
\end{tabular}\\
For further symbols and acronyms that are prevalant
throughout the project please reference the SRS \citep{SRS}.


\wss{symbols, abbreviations, or acronyms --- you can simply reference the SRS
  \citep{SRS} tables, if appropriate}

\wss{Remove this section if it isn't needed}

\newpage

\pagenumbering{arabic}

This document is the system Verification and Validation plan (Syst VnV plan). This document introduces
the plan to verify the correctness and the validity of the software being developed.  For the VnV plan we
will state the FR and the NFR, and try to map out a plan to successfully achieve success on all necessary 
requirements. We will state all necessary details in this document for a reader to be able to map out and 
build their own system tests for the corresponding software. The primary objective this document seeks to
accomplish is state a plan that can test the compliance with the NFR and FR (verification). Also, to ensure 
that the software will meet the expectations of users (validation). Since this is a library, further a 
numerical library, the primary tests will be accuracy of the software and reliability to give relatively 
accurate solutions to users' specification. 
\dots... \wss{provide an introductory blurb and roadmap of the
  Verification and Validation plan}

\section{General Information}

\subsection{Summary}
The main project 
as discussed in the SRS \citep{SRS}, is to build a library of function optimizers. Specifically, we will
focus on function minimization and restrict our library to a few solvers (see SRS). We will focus on the 
David-Fletcher-Powell (DFP), Fletcher-Reeves conjugate gradient (FRCG), and Broyden-Fletcher-Goldfarb-Shanno (BFGS) algorithms. 
These methods are split between two classes of function optimizers; Conjugate-Gradient (Fletcher-Reeves) and Quasi-Newton 
methods (DFP and BFGS). Both of these classes still share the common idea of a line search. For further detail on the families
of solvers and optimization please reference the SRS \citep{SRS}.
\\

Our library will be used in the following manner; a user will call a specific minimizer listed prior, the user will provide the 
function required to be minimized (in the appropriate format) with the specified parameters for the minimizer. The minimizer will
return a solution based off the parameters given. 
\wss{Say what software is being tested.  Give its name and a brief overview of
  its general functions.}

\subsection{Objectives}

It is important to note these optimizers are not oracles; as such we do not
expect them to return the correct global minimum each time. The `correct' solution we seek is the output expected from running
the specific optimizer with the specified parameters. Therefore, the expected stated primary objective of finding a `correct' 
solution comes with some caveats. We will therefore prioritize our libraries accuracy to their respective algorithms. We wish
to build a library of function minimizers that faithfully follow the computation of the original algorithm design, even if that
is an incorrect solution.  To summarize we wish to build confidence in the software's faithfulness to the original algorithm's 
expected solutions. We will also expect the user to be able to call the function and the library to execute several modules working
in unison without a hitch. 
We will also wish to demonstrate usability for the expected users of this software. We will not ensure this is the most efficient 
realization of these algorithms; as this is out of the scope of this project. There are numerous high-computing libaries that have
created the algorithms we are building and we do not wish to compete with them. We will also expect that the external libraries from 
which we will verify our solutions' `correctness' are reliable and have been thoroughly tested. We will also hold this standard for
the textbooks from which we will have test cases for our library. 


\wss{State what is intended to be accomplished.  The objective will be around
  the qualities that are most important for your project.  You might have
  something like: ``build confidence in the software correctness,''
  ``demonstrate adequate usability.'' etc.  You won't list all of the qualities,
  just those that are most important.}

\wss{You should also list the objectives that are out of scope.  You don't have 
the resources to do everything, so what will you be leaving out.  For instance, 
if you are not going to verify the quality of usability, state this.  It is also 
worthwhile to justify why the objectives are left out.}

\wss{The objectives are important because they highlight that you are aware of 
limitations in your resources for verification and validation.  You can't do everything, 
so what are you going to prioritize?  As an example, if your system depends on an 
external library, you can explicitly state that you will assume that external library 
has already been verified by its implementation team.}

\subsection{Relevant Documentation}
The SRS, MG, and MIS are relevant documents to the VnV and will encompass the complete
documentation for the Optimization Library. The SRS will contain a high-level explanation
of the purpose, execution, and ideas behind this project. It will also define common themes
between the different minimizers, and give a high-level walkthrough of function optimization. 
This document is great for understanding certain design decisions and the purpose of each module.
The MG and MIS documents are important to illustrate the purpose of each module and how they
interact with each other. These documents coupled with the VnV should allow a reader to go through 
the whole software development process for this project.
\wss{Reference relevant documentation.  This will definitely include your SRS
  and your other project documents (design documents, like MG, MIS, etc).  You
  can include these even before they are written, since by the time the project
  is done, they will be written.}

\citet{SRS}

\wss{Don't just list the other documents.  You should explain why they are relevant and 
how they relate to your VnV efforts.}

\section{Plan}
In this section we provide a roadmap for our VnV. We wish to ensure the valdity of our work
as such we will have a team reviewing our documentation ensuring our ideas are correct and in 
line with the goals we seek to acheive \citep{SRS}. We will also introduce multiple tests that 
are expansive enough to give confidence to future users that our library will do what we expect
it to do. 
\wss{Introduce this section.   You can provide a roadmap of the sections to
  come.}

\subsection{Verification and Validation Team}

\begin{itemize}
  \item Author: Fasil Cheema
  \item Primary Reviewer: Morteza Mirzaei
  \item Secondary Reviewer: Dr Spencer Smith
  \item Secondary Reviewer: Xinyu
\end{itemize}
\wss{Your teammates.  Maybe your supervisor.
  You should do more than list names.  You should say what each person's role is
  for the project's verification.  A table is a good way to summarize this information.}

\subsection{SRS Verification Plan}
For the SRS verification plan a critical component will be taking criticisms from the 
reviewers of the SRS. We will implement these criticisms into our updated document. 
We will also have a checklist for the SRS to ensure we maintain the quality of the document. 
This checklist will contain information to check over grammatical errors such as spelling mistakes,
grammar etc. We will also ensure the checklist covers major concepts we should have in our SRS such 
as proper documentation of the high level concept of the convex optimization.

\wss{List any approaches you intend to use for SRS verification.  This may include
  ad hoc feedback from reviewers, like your classmates, or you may plan for 
  something more rigorous/systematic.}

\wss{Maybe create an SRS checklist?}

\subsection{Design Verification Plan}

\wss{Plans for design verification}

This section introduces the plan for design verification. For this plan we will have 
team members reviewing the high level calculations of the functions are correct.
Domain experts are going to ensure these calculations are line with the original 
algorithms.

\wss{The review will include reviews by your classmates}

\wss{Create a checklists?}

\subsection{Verification and Validation Plan Verification Plan}

We will ensure the validity of the plan by checking with our reviewers and
their feedback. The VnV feedback will be implemented into the plan accordingly.
We will also have a checklist to ensure the VnV plan covers all bases from minor 
things such as spelling quality, to major high level tests that need to be verified.
\wss{The verification and validation plan is an artifact that should also be
verified. Techniques for this include review and mutation testing.}

\wss{The review will include reviews by your classmates}

\wss{Create a checklists?}

\subsection{Implementation Verification Plan}

The verification plan will be done by testing all the FR and
NFR. The tests can be found in section 4. In addition, we will check over our code built
in Python with a linter; PyLint. We will also have a code inspection and do rubber duck
testing. We will also conduct unit testing for modules 
within the testing scope. Details for unit testing can be found in section 4.

\wss{You should at least point to the tests listed in this document and the unit
  testing plan.}

\wss{In this section you would also give any details of any plans for static
  verification of the implementation.  Potential techniques include code
  walkthroughs, code inspection, static analyzers, etc.}

\subsection{Automated Testing and Verification Tools}
In this section we will use several tools for automated testing. We will use the PyTest
 (Unittest) framework. This will allow multiple unit tests to be created and conducted.
 We will also employ PyLint as the linter of choice for ensuring we avoid unnecessary
 mistakes in the code. 
\wss{What tools are you using for automated testing.  Likely a unit testing
  framework and maybe a profiling tool, like ValGrind.  Other possible tools
  include a static analyzer, make, continuous integration tools, test coverage
  tools, etc.  Explain your plans for summarizing code coverage metrics.
  Linters are another important class of tools.  For the programming language
  you select, you should look at the available linters.  There may also be tools
  that verify that coding standards have been respected, like flake9 for
  Python.}

\wss{If you have already done this in the development plan, you can point to
that document.}

\wss{The details of this section will likely evolve as you get closer to the
  implementation.}

\subsection{Software Validation Plan}
There is a textbook for convex optimization problems \citep{Boyd2006} where known
solutions for their respective algorithms are known. This will be useful to verify 
if our algorithm will output the correct solution when given certain parameters.
There are also libraries that have trusted builds of the minimizers we seek to make. 
Scipy comes built in with a minimizer function that allows us to use any specific
minimizer we want including the ones presented in this doc (DFP,BFGS, FRCG).
\wss{If there is any external data that can be used for validation, you should
  point to it here.  If there are no plans for validation, you should state that
  here.}

\wss{You might want to use review sessions with the stakeholder to check that
the requirements document captures the right requirements.  Maybe task based
inspection?}

\wss{For those capstone teams with an external supervisor, the Rev 0 demo should 
be used as an opportunity to validate the requirements.  You should plan on 
demonstrating your project to your supervisor shortly after the scheduled Rev 0 demo.  
The feedback from your supervisor will be very useful for improving your project.}

\wss{For teams without an external supervisor, user testing can serve the same purpose 
as a Rev 0 demo for the supervisor.}

\wss{This section might reference back to the SRS verification section.}

\section{System Test Description}

\subsection{Tests for Functional Requirements}
Functional Tests are given for this document in the SRS \citep{SRS}. We will have input 
We will have input tests for our functional requirements in \citep{SRS}.
\wss{Subsets of the tests may be in related, so this section is divided into
  different areas.  If there are no identifiable subsets for the tests, this
  level of document structure can be removed.}

\wss{Include a blurb here to explain why the subsections below
  cover the requirements.  References to the SRS would be good here.}
\subsubsection{Input}
These tests will attempt to verify the following requirements also found in the SRS:
\begin{itemize}
  \item R1: The function and input vector are in the specified dimension which does not exceed
  maximum defined dimension.
  \item R2: The function will notify user of invalid data size (matrix shape mismatch)
\end{itemize}
\subsubsection{Area of Testing1-Functional Requirements 1,2}
FR 1 states that the matrices, and vectors in a function we seek to minimize should all be
of appropriate size. That is, we should not have an error when attempting to conduct matrix 
operations. Consequently, FR2 states that the function shall notify the user in the case where 
this occurs; a size mismatch. This section will cover the area of testing related to these 2 FR. 
\wss{It would be nice to have a blurb here to explain why the subsections below
  cover the requirements.  References to the SRS would be good here.  If a section
  covers tests for input constraints, you should reference the data constraints
  table in the SRS.}
		
\paragraph{Tests for FR1, FR2}

\begin{enumerate}

\item{test-Default, non-problematic, inputs\\}

Control: Automatic
					
Initial State: Pending  
					
Input:for matrix $\mathbf{A}$, vector $\mathbf{\vec{b}}$, scalar $c$
We have for $\mathbf{A}$: this matrix will be the identity matrix ranging 
from dimensions 1 to our max dimension 6. We will have $\mathbf{\vec{b}}$
also be a vector of 1s ranging from dimensions 1 to 6 as well. Finally we 
will have the scalar $c$ be set to 1 for all the tests. 
					
Output: Valid (True)  

Test Case Derivation: The size mismatch detector should not have an issue accepting these
inputs. They should all pass. \wss{Justify the expected value given in the Output field}
					
How test will be performed: Utilizing PyTest we will set the unit tests with the
appropriate dimensions as specified above. 
					
\item{test-Problematic Input\\}

Control: Automatic
					
Initial State: Pending
					
Input: for matrix $\mathbf{A}$, vector $\mathbf{\vec{b}}$, scalar $c$
We have for $\mathbf{A}$: this matrix will be the identity matrix ranging 
from dimensions 1 to our max dimension 6. We will have $\mathbf{\vec{b}}$
also be a vector of 1s ranging from dimensions 1 to 6 as well. Finally we 
will have the scalar $c$ be set to 1 for all the tests. For this test we 
will not have the dimension of $\mathbf{A}$ and $\mathbf{\vec{b}}$ be the same
but we will ensure they are always different. In other words we will try all 
possible permutations of dimensions for both the matrix and vector where the
dimensions of both are not equal.
					
Output: The size mismatch detector should detect a problem and raise an error. \wss{The expected result for the given inputs}

Test Case Derivation: This is an invalid size and further in the library there will be invalid 
matrix operations (cannot do matrix multiplication for matrices of invalid sizes). \wss{Justify the expected value given in the Output field}

How test will be performed: Utilizing PyTest we will set the unit tests with the
appropriate dimensions as specified above.

\end{enumerate}

\subsection{Tests for Nonfunctional Requirements}
The main test for the NFR will be related to accuracy. We will like to ensure that 
our solution from the algorithm of choice is within the specified accuracy parameter 
to the solution from the corresponding trusted solution.
\wss{The nonfunctional requirements for accuracy will likely just reference the
  appropriate functional tests from above.  The test cases should mention
  reporting the relative error for these tests.  Not all projects will
  necessarily have nonfunctional requirements related to accuracy}

\wss{Tests related to usability could include conducting a usability test and
  survey.  The survey will be in the Appendix.}

\wss{Static tests, review, inspections, and walkthroughs, will not follow the
format for the tests given below.}

\subsubsection{Area of Testing1}
		
\paragraph{PSD Test 1 Full step }

\begin{enumerate}

\item{PSD Test 1 Full step}\

Type: Automatic
					
Initial State: Pending
					
Input/Condition: $\mathbf{A}$ will be the identity matrix ranging from
1 to 6 dimensions. $\mathbf{\vec{b}}$ will be a vector of 1s ranging from 1 
6 dimensions along with $\mathbf{A}$. The scalar $c$ will be set to 1. We will
use one of each algorithm (DFP, FRCG, and BFGS) with a full step (step size set to 1),
an initial starting choice of the zero vector for the corresponding dimension. For the 
quasi-newton methods we will have an initial Hessian, $H_{0}$, set tot be the identity 
matrix of the corresponding dimensionality. 
					
Output/Result: Relative Error of the two vectors (utilizing the norm of a vector). 
					
How test will be performed: We will compute our respective algorithms then conduct the same 
computation via our trusted source (scipy.minmize('bfgs')...) and then compute the relative 
error. This is done by taking the norm of the difference of the two vectors over the norm of
the `true' solution. If this is below the threshold we specify: accuracy parameter ($\epsilon_{acc} = 1\%$) 


\item{PSD Test 2 adaptive Step}\

Type: Automatic
					
Initial State: Pending
					
Input/Condition: $\mathbf{A}$ will be the identity matrix ranging from
1 to 6 dimensions. $\mathbf{\vec{b}}$ will be a vector of 1s ranging from 1 
6 dimensions along with $\mathbf{A}$. The scalar $c$ will be set to 1. We will
use one of each algorithm (DFP, FRCG, and BFGS) with an adaptive step size,
an initial starting choice of the zero vector for the corresponding dimension. For the 
quasi-newton methods we will have an initial Hessian, $H_{0}$, set tot be the identity 
matrix of the corresponding dimensionality. In this case the adaptive step size requires 
the individual algorithm to calculate the step size at each iteration. This is catered for 
each individual algorithm and adds another degree of complexity.
					
Output/Result: Relative Error of the two vectors (utilizing the norm of a vector). 
					
How test will be performed: We will compute our respective algorithms then conduct the same 
computation via our trusted source (scipy.minmize('bfgs')...) and then compute the relative 
error. This is done by taking the norm of the difference of the two vectors over the norm of
the `true' solution. If this is below the threshold we specify: accuracy parameter ($\epsilon_{acc} = 1\%$) 

\item{non-PSD Test 2 adaptive Step}\

Type: Automatic
					
Initial State: Pending
					
Input/Condition: $\mathbf{A}$ will be the identity matrix ranging from
1 to 6 dimensions, except the first entry will be $-1000$. This makes the matrix non-PSD
this is to violate the assumptions of our algorithms which require a PSD matrix to converge 
to a global minima. $\mathbf{\vec{b}}$ will be a vector of 1s ranging from 1 
6 dimensions along with $\mathbf{A}$. The scalar $c$ will be set to 1. We will
use one of each algorithm (DFP, FRCG, and BFGS) with an adaptive step size,
an initial starting choice of the zero vector for the corresponding dimension. For the 
quasi-newton methods we will have an initial Hessian, $H_{0}$, set tot be the identity 
matrix of the corresponding dimensionality. In this case the adaptive step size requires 
the individual algorithm to calculate the step size at each iteration. This is catered for 
each individual algorithm and adds another degree of complexity.
					
Output/Result: Relative Error of the two vectors (utilizing the norm of a vector). 
					
How test will be performed: We will compute our respective algorithms then conduct the same 
computation via our trusted source (scipy.minmize('bfgs')...) and then compute the relative 
error. This is done by taking the norm of the difference of the two vectors over the norm of
the `true' solution. If this is below the threshold we specify: accuracy parameter ($\epsilon_{acc} = 1\%$) 

\item{Portability Test 1- Linux\\}

Type: Manual 
					
Initial State: Pending
					
Input: 
					
Output: 
					
How test will be performed: Using the conda environment, on a fresh build of ubuntu 22.0
we wish to first install the environment then run the library with the first test for the 
NFR (PSD Test 1 Full step).

\item{Portability Test 2- Windows\\}

Type: Manual 
					
Initial State: Pending
					
Input: 
					
Output: 
					
How test will be performed: Using the conda environment, on windows 11
we wish to first install the environment then run the library with the first test for the 
NFR (PSD Test 1 Full step).

\end{enumerate}

\subsubsection{Area of Testing2}

...

\subsection{Traceability Between Test Cases and Requirements}
\begin{table}[h!]
  \centering
  \begin{tabular}{|c|c|c|c|c|c|c|c|c|c|c|c|c|c|c|c|c|c|c|c|c|c|c|c|}
  \hline        
    & FR1& FR2 & NFR1 & NFR2& NFR3 & NFR4& NFR5\\
  \hline
  TM:CON                    &X & & & & & & &  \\ \hline
  GD\ref{gd:funcmin}        &X & & & & & & &  \\ \hline
  DD\ref{dd:DotProd}      & &X & & & & & &  \\ \hline
  DD\ref{dd:Transpose} & & &X & & & & &  \\ \hline
  DD\ref{dd:Gradient}  & & & &X & & & &  \\ \hline
  DD\ref{dd:Hessian}    & & & & & X& & &  \\ \hline
  IM\ref{im:linesearch}     & &X &X &X & &X & &  \\ \hline
  IM\ref{im:CGsearchdirection}      & &X &X &X & & & X&  \\ \hline
  IM\ref{im:QNsearchdirection}      & &X &X &X &X & & &X  \\ \hline
  
  \end{tabular}
  \caption{Traceability Matrix Showing the Connections Between Items of Different Sections}
  \label{Table:trace}
  \end{table}
\wss{Provide a table that shows which test cases are supporting which
  requirements.}

\section{Unit Test Description}

\wss{This section should not be filled in until after the MIS (detailed design
  document) has been completed.}

\wss{Reference your MIS (detailed design document) and explain your overall
philosophy for test case selection.}  

\wss{To save space and time, it may be an option to provide less detail in this section.  
For the unit tests you can potentially layout your testing strategy here.  That is, you 
can explain how tests will be selected for each module.  For instance, your test building 
approach could be test cases for each access program, including one test for normal behaviour 
and as many tests as needed for edge cases.  Rather than create the details of the input 
and output here, you could point to the unit testing code.  For this to work, you code 
needs to be well-documented, with meaningful names for all of the tests.}

\subsection{Unit Testing Scope}

\wss{What modules are outside of the scope.  If there are modules that are
  developed by someone else, then you would say here if you aren't planning on
  verifying them.  There may also be modules that are part of your software, but
  have a lower priority for verification than others.  If this is the case,
  explain your rationale for the ranking of module importance.}

\subsection{Tests for Functional Requirements}

\wss{Most of the verification will be through automated unit testing.  If
  appropriate specific modules can be verified by a non-testing based
  technique.  That can also be documented in this section.}

\subsubsection{Module 1}

\wss{Include a blurb here to explain why the subsections below cover the module.
  References to the MIS would be good.  You will want tests from a black box
  perspective and from a white box perspective.  Explain to the reader how the
  tests were selected.}

\begin{enumerate}

\item{test-id1\\}

Type: \wss{Functional, Dynamic, Manual, Automatic, Static etc. Most will
  be automatic}
					
Initial State: 
					
Input: 
					
Output: \wss{The expected result for the given inputs}

Test Case Derivation: \wss{Justify the expected value given in the Output field}

How test will be performed: 
					
\item{test-id2\\}

Type: \wss{Functional, Dynamic, Manual, Automatic, Static etc. Most will
  be automatic}
					
Initial State: 
					
Input: 
					
Output: \wss{The expected result for the given inputs}

Test Case Derivation: \wss{Justify the expected value given in the Output field}

How test will be performed: 

\item{...\\}
    
\end{enumerate}

\subsubsection{Module 2}

...

\subsection{Tests for Nonfunctional Requirements}

\wss{If there is a module that needs to be independently assessed for
  performance, those test cases can go here.  In some projects, planning for
  nonfunctional tests of units will not be that relevant.}

\wss{These tests may involve collecting performance data from previously
  mentioned functional tests.}

\subsubsection{Module ?}
		
\begin{enumerate}

\item{test-id1\\}

Type: \wss{Functional, Dynamic, Manual, Automatic, Static etc. Most will
  be automatic}
					
Initial State: 
					
Input/Condition: 
					
Output/Result: 
					
How test will be performed: 
					
\item{test-id2\\}

Type: Functional, Dynamic, Manual, Static etc.
					
Initial State: 
					
Input: 
					
Output: 
					
How test will be performed: 

\end{enumerate}

\subsubsection{Module ?}

...

\subsection{Traceability Between Test Cases and Modules}

\wss{Provide evidence that all of the modules have been considered.}
				
\bibliographystyle{plainnat}

\bibliography{../../refs/References}

\newpage

\section{Appendix}

This is where you can place additional information.

\subsection{Symbolic Parameters}

The definition of the test cases will call for SYMBOLIC\_CONSTANTS.
Their values are defined in this section for easy maintenance.

\subsection{Usability Survey Questions?}

\wss{This is a section that would be appropriate for some projects.}

\newpage{}
\section*{Appendix --- Reflection}

The information in this section will be used to evaluate the team members on the
graduate attribute of Lifelong Learning.  Please answer the following questions:

\newpage{}
\section*{Appendix --- Reflection}

\wss{This section is not required for CAS 741}

The information in this section will be used to evaluate the team members on the
graduate attribute of Lifelong Learning.  Please answer the following questions:

\begin{enumerate}
  \item What knowledge and skills will the team collectively need to acquire to
  successfully complete the verification and validation of your project?
  Examples of possible knowledge and skills include dynamic testing knowledge,
  static testing knowledge, specific tool usage etc.  You should look to
  identify at least one item for each team member.
  \item For each of the knowledge areas and skills identified in the previous
  question, what are at least two approaches to acquiring the knowledge or
  mastering the skill?  Of the identified approaches, which will each team
  member pursue, and why did they make this choice?
\end{enumerate}

\end{document}