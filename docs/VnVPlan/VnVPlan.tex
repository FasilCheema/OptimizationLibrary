\documentclass[12pt, titlepage]{article}

\usepackage{booktabs}
\usepackage{tabularx}
\usepackage{hyperref}
\hypersetup{
    colorlinks,
    citecolor=blue,
    filecolor=black,
    linkcolor=red,
    urlcolor=blue
}
\usepackage[round]{natbib}

%added by my Fasil
\usepackage{mathtools}
\usepackage{bm}
\usepackage{esvect}

%% Comments

\usepackage{color}

\newif\ifcomments\commentstrue %displays comments
%\newif\ifcomments\commentsfalse %so that comments do not display

\ifcomments
\newcommand{\authornote}[3]{\textcolor{#1}{[#3 ---#2]}}
\newcommand{\todo}[1]{\textcolor{red}{[TODO: #1]}}
\else
\newcommand{\authornote}[3]{}
\newcommand{\todo}[1]{}
\fi

\newcommand{\wss}[1]{\authornote{blue}{SS}{#1}} 
\newcommand{\plt}[1]{\authornote{magenta}{TPLT}{#1}} %For explanation of the template
\newcommand{\an}[1]{\authornote{cyan}{Author}{#1}}

%% Common Parts

\newcommand{\progname}{ProgName} % PUT YOUR PROGRAM NAME HERE
\newcommand{\authname}{Team \#, Team Name
\\ Student 1 name
\\ Student 2 name
\\ Student 3 name
\\ Student 4 name} % AUTHOR NAMES                  

\usepackage{hyperref}
    \hypersetup{colorlinks=true, linkcolor=blue, citecolor=blue, filecolor=blue,
                urlcolor=blue, unicode=false}
    \urlstyle{same}
                                


\begin{document}

\title{Optimization Library: System Verification and Validation Plan} 

\author{Fasil Cheema}
\date{Feb 19, 2024}
	
\maketitle

\pagenumbering{roman}

\section*{Revision History}

\begin{tabularx}{\textwidth}{p{3cm}p{2cm}X}
\toprule {\bf Date} & {\bf Version} & {\bf Notes}\\
\midrule
Feb 19, 2024 & 1.0 & Initial Upload\\
April 15, 2024 & 2.0 & Revised Version\\
\bottomrule
\end{tabularx}

~\\


\newpage

\tableofcontents

\newpage

\section{Symbols, Abbreviations, and Acronyms}

\renewcommand{\arraystretch}{1.2}
\begin{tabular}{l l} 
  \toprule		
  \textbf{symbol} & \textbf{description}\\
  \midrule 
  T & Test\\
  VnV & Verification and Validation\\
  FR & Functional Requirement\\
  NFR & Non-Functional Requirement\\
  MG & Module Guide\\
  MIS & Module Interface Specification\\
  \bottomrule
\end{tabular}\\
For further symbols and acronyms that are prevalant
throughout the project please reference the SRS \citep{SRS}.

\newpage

\pagenumbering{arabic}

This document is the system Verification and Validation plan (Syst VnV plan). This document introduces
the plan to verify the correctness and the validity of the software being developed.  For the VnV plan we
will state the FR and the NFR, and try to map out a plan to successfully achieve success on all necessary 
requirements. We will state all necessary details in this document for a reader to be able to map out and 
build their own system tests for the corresponding software. The primary objective this document seeks to
accomplish is state a plan that can test the compliance with the NFR and FR (verification). Also, to ensure 
that the software will meet the expectations of users (validation). Since this is a library, further a 
numerical library, the primary tests will be accuracy of the software and reliability to give relatively 
accurate solutions to users' specification. 


\section{General Information}
This section provides general information about the OptLib project.

\subsection{Summary}
The main project as discussed in the SRS \citep{SRS}, is to build a library of function optimizers. Specifically, we will
focus on function minimization and restrict our library to a few solvers (see SRS). We will focus on the 
David-Fletcher-Powell (DFP), Fletcher-Reeves conjugate gradient (FRCG), and Broyden-Fletcher-Goldfarb-Shanno (BFGS) algorithms. 
These methods are split between two classes of function optimizers; Conjugate-Gradient (Fletcher-Reeves) and Quasi-Newton 
methods (DFP and BFGS). Both of these classes still share the common idea of a line search. For further detail on the families
of solvers and optimization please reference the SRS \citep{SRS}.
\\

Our library will be used in the following manner; a user will call a specific minimizer listed prior, the user will provide the 
function required to be minimized (in the appropriate format) with the specified parameters for the minimizer. The minimizer will
return a solution based off the parameters given. 

\subsection{Objectives}

It is important to note these optimizers are not oracles; as such we do not
expect them to return the correct global minimum each time. The `correct' solution we seek is the output expected from running
the specific optimizer with the specified parameters. Therefore, the expected stated primary objective of finding a `correct' 
solution comes with some caveats. We will therefore prioritize our libraries accuracy to their respective algorithms. We wish
to build a library of function minimizers that faithfully follow the computation of the original algorithm design, even if that
is an incorrect solution.  To summarize we wish to build confidence in the software's faithfulness to the original algorithm's 
expected solutions. We will also expect the user to be able to call the function and the library to execute several modules working
in unison without a hitch. 
We will also wish to demonstrate usability for the expected users of this software. We will not ensure this is the most efficient 
realization of these algorithms; as this is out of the scope of this project. There are numerous high-computing libaries that have
created the algorithms we are building and we do not wish to compete with them. We will also expect that the external libraries from 
which we will verify our solutions' `correctness' are reliable and have been thoroughly tested. We will also hold this standard for
the textbooks from which we will have test cases for our library. 


\subsection{Relevant Documentation}
The SRS, MG, and MIS are relevant documents to the VnV and will encompass the complete
documentation for the Optimization Library \citep{SRS}. The SRS will contain a high-level explanation
of the purpose, execution, and ideas behind this project. It will also define common themes
between the different minimizers, and give a high-level walkthrough of function optimization. 
This document is great for understanding certain design decisions and the purpose of each module.
The MG and MIS documents are important to illustrate the purpose of each module and how they
interact with each other. These documents coupled with the VnV should allow a reader to go through 
the whole software development process for this project.



\section{Plan}
In this section we provide a roadmap for our VnV. We wish to ensure the valdity of our work
as such we will have a team reviewing our documentation ensuring our ideas are correct and in 
line with the goals we seek to achieve \citep{SRS}. We will also introduce multiple tests that 
are expansive enough to give confidence to future users that our library will do what we expect
it to do. 

\subsection{Verification and Validation Team}

\begin{itemize}
  \item Author: Fasil Cheema
  \item Primary Reviewer: Morteza Mirzaei
  \item Secondary Reviewer: Dr Spencer Smith
  \item Secondary Reviewer (SRS): Nada
  \item Secondary Reviewer (VnV): Xinyu
  \item Secondary Reviewer (MG/MIS): Valerie
\end{itemize}

\subsection{SRS Verification Plan}
For the SRS verification plan a critical component will be taking criticisms from the 
reviewers of the SRS. We will implement these criticisms into our updated document. 
We will also maintain the quality of the document. 
We will ensure that we check over grammatical errors such as spelling mistakes,
grammar etc. We will also ensure we cover major concepts we should have in our SRS such 
as proper documentation of the high level concept of the convex optimization.

\subsection{Design Verification Plan}


This section introduces the plan for design verification. For this plan we will have 
team members reviewing the high level calculations of the functions are correct.
Domain experts (primary reviewer Morteza) is going to ensure these calculations are line with the original 
algorithms.

\subsection{Verification and Validation Plan Verification Plan}

We will ensure the validity of the plan by checking with our reviewers and
their feedback. The VnV feedback will be implemented into the plan accordingly.
We will also have a checklist to ensure the VnV plan covers all bases from minor 
things such as spelling quality, to major high level tests that need to be verified.

\subsection{Implementation Verification Plan}

The verification plan will be done by testing all the FR and
NFR. The tests can be found in section 4. In addition, we will check over our code built
in Python with a linter; PyLint. This checks for incorrect code such as incorrect function calls. It can also check for code that ends up not being run or overridden. PyLint also ensures we keep pythonic code this is very good for code readability. We will also have a code inspection and do rubber duck
testing. We will also conduct unit testing for modules 
within the testing scope.  This testing will be done by the author by running the PyLint over the code. The author will ensure PyLint does not have any remaining issues (at least ones that cause issues). The author will ensure that PyLint makes sure there is no unused code, incorrect function calls, packages that are not imported, misspelled variables or functions, and any other mistakes in the code. Details for unit testing can be found in section 4.


\subsection{Automated Testing and Verification Tools}
In this section we will use several tools for automated testing. We will use the PyTest
 (Unittest) framework. This will allow multiple unit tests to be created and conducted.
 We will also employ PyLint as the linter of choice for ensuring we avoid unnecessary
 mistakes in the code. We will also have continuous integration where each module will be individually tested and validated. Then everything comes together and is tested.


\subsection{Software Validation Plan}
There is a textbook for convex optimization problems \citep{Boyd2006} where known
solutions for their respective algorithms are known. This will be useful to verify 
if our algorithm will output the correct solution when given certain parameters.
There are also libraries that have trusted builds of the minimizers we seek to make. 
Scipy comes built in with a minimizer function that allows us to use any specific
minimizer we want including the ones presented in this doc (DFP,BFGS, FRCG).


\section{System Test Description}
This section will contain information on the system tests.

\subsection{Tests for Functional Requirements}
Functional Tests are given for this document in the SRS \citep{SRS}. We will have input 
We will have input tests for our functional requirements in \citep{SRS}.

\subsubsection{Input}
These tests will attempt to verify the following requirements also found in the SRS:
\begin{itemize}
  \item FR1: The function and input vector are in the specified dimension which does not exceed
  maximum defined dimension.
  \item FR2: The function can be represented in the quadratic form (ie valid parameters)
\end{itemize}
\subsubsection{Area of Testing1-Functional Requirements 1,2}
FR 1 states that the matrices, and vectors in a function we seek to minimize should all be
of appropriate size. That is, we should not have an error when attempting to conduct matrix 
operations. Consequently, FR2 states that the function shall be in the quadratic form. This means the input should be not only in the appropriate dimensionality but also be valid (the format must be a numpy array of 2 dimensions for vectors etc)  This section will cover the area of testing related to these 2 FR. 

		
\paragraph{Tests for FR1, FR2}

\begin{enumerate}

\item{test-Default, non-problematic, inputs\\}

Control: Automatic
					
Initial State: Pending  
					
Input:for matrix $\mathbf{A}$, vector $\mathbf{\vec{b}}$, scalar $c$
We have for $\mathbf{A}$: this matrix will be the identity matrix ranging 
from dimensions 1 to our max dimension 6. We will have $\mathbf{\vec{b}}$
also be a vector of 1s ranging from dimensions 1 to 6 as well. Finally we 
will have the scalar $c$ be set to 1 for all the tests. 
					
Output: Valid (True)  

Test Case Derivation: The size mismatch detector should not have an issue accepting these
inputs. They should all pass. 
					
How test will be performed: Import the input verification module and check different values for the input parameters $\mathbf{A}$, $\vv{b}$, $c$, $\vv{x}_0$, $\mathbf{H}_0$, $\mathbf{B}_0$, step\_size, max\_s, min\_err.
\\

\noindent Check for valid $\mathbf{A}$ (2 dimensional matrix, a square matrix, a numpy array of floats, a numpy array).  
\\

\noindent Check for valid $\vv{b}$ (2 dimensional matrix, a row vector of appropriate shape, a numpy array of floats, a numpy array, should match dimensionality of $\mathbf{A}$).
\\

\noindent Check for valid $c$ (a valid float or integer).
\\

\noindent Check for valid $\vv{x}_0$ (a 2 dimensional matrix, a column vector of appropriate shape, a numpy array of floats, a numpy array, should match dimensionality of $\mathbf{A}$).
\\

\noindent Check for valid $\mathbf{H}_0$ (2 dimensional matrix, a square matrix, a numpy array of floats, a numpy array, appropriate dimensionality size matching $\mathbf{A}$).  
\\

\noindent Check for valid $\mathbf{B}_0$ (2 dimensional matrix, a square matrix, a numpy array of floats, a numpy array, appropriate dimensionality size matching $\mathbf{A}$).  
\\

\noindent Check for valid step\_size (a natural number also checks if below the max number of steps defined in \citep{SRS}).  
\\

\noindent Check for valid max\_s (a natural number also checks if below the max number of steps defined in \citep{SRS}).  
\\

\noindent Check for valid min\_err (between 0 and 1 and also above the min threshold defined in \citep{SRS}).  
\\
					
\item{test-Problematic Input\\}

Control: Automatic
					
Initial State: Pending
					
Input: for matrix $\mathbf{A}$, vector $\mathbf{\vec{b}}$, scalar $c$
We have for $\mathbf{A}$: this matrix will be the identity matrix ranging 
from dimensions 1 to our max dimension 6. We will have $\mathbf{\vec{b}}$
also be a vector of 1s ranging from dimensions 1 to 6 as well. Finally we 
will have the scalar $c$ be set to 1 for all the tests. For this test we 
will not have the dimension of $\mathbf{A}$ and $\mathbf{\vec{b}}$ be the same
but we will ensure they are always different. In other words a test for if the matrices are not 2 dimensional matrices, the vectors are not column or row vectors specifically, c is not a scalar, or if matrix dimensions are inconcisent or if an input vector/matrix dimensions are not consistent.
					
Output: The input verification module should detect a problem and raise an error.

Test Case Derivation: This is an invalid size and further in the library there will be invalid 
matrix operations (cannot do matrix multiplication for matrices of invalid sizes).

How test will be performed: Import the input verification module and check different values for the input parameters $\mathbf{A}$, $\vv{b}$, $c$, $\vv{x}_0$, $\mathbf{H}_0$, $\mathbf{B}_0$, step\_size, max\_s, min\_err.
\\

\noindent Check for invalid $\mathbf{A}$ (not 2 dimensional matrix, not a square matrix, not a numpy array of floats, not a numpy array).  
\\

\noindent Check for invalid $\vv{b}$ (not 2 dimensional matrix, not a row vector of appropriate shape, not a numpy array of floats, not a numpy array, should match dimensionality of $\mathbf{A}$).
\\

\noindent Check for invalid $c$ (not a valid float or integer).
\\

\noindent Check for invalid $\vv{x}_0$ (not 2 dimensional matrix, not a column vector of appropriate shape, not a numpy array of floats, not a numpy array, should match dimensionality of $\mathbf{A}$).
\\

\noindent Check for invalid $\mathbf{H}_0$ (not 2 dimensional matrix, not a square matrix, not a numpy array of floats, not a numpy array, not appropriate dimensionality size matching $\mathbf{A}$).  
\\

\noindent Check for invalid $\mathbf{B}_0$ (not 2 dimensional matrix, not a square matrix, not a numpy array of floats, not a numpy array, not appropriate dimensionality size matching $\mathbf{A}$).  
\\

\noindent Check for invalid step\_size (not a natural number also checks if above the max number of steps defined in \citep{SRS}).  
\\

\noindent Check for invalid max\_s (not a natural number also checks if above the max number of steps defined in \citep{SRS}).  
\\

\noindent Check for invalid min\_err (not between 0 and 1 and also below the min threshold defined in \citep{SRS}).  
\\

\end{enumerate}

\subsection{Tests for Nonfunctional Requirements}
The main test for the NFR will be related to accuracy. We will like to ensure that 
our solution from the algorithm of choice is within the specified accuracy parameter 
to the solution from the corresponding trusted solution. Wherever the `textbook' is referenced it is in reference to the following text \citep{Boyd2006}.

\subsubsection{Area of Testing NFR}
		
\paragraph{PSD Test 1 Exact step }

\begin{enumerate}

\item{PSD Test 1 Full step}\

Type: Manual 
					
Initial State: Pending
					
Input/Condition: $\mathbf{A}$ will be set to a matrix specified.
$\mathbf{\vec{b}}$ will be set to a vector specified. The scalar $c$ will be set to 0. We will
use one of each algorithm (DFP, FRCG, and BFGS) with a full step (step size set to 1),
an initial starting choice of the zero vector for the corresponding dimension. For the 
quasi-newton methods we will have an initial Hessian, $H_{0}$ or inverse Hessian, $B_{0}$ set to be the identity 
matrix of the corresponding dimensionality. 
					
Output/Result: Relative Error of the two vectors (utilizing the norm of a vector). The result will be from a textbook and will correspond to the known value. Final result is a Pass/Fail depending on if the norm value is below the err threshold.
					
How test will be performed: We will compute our respective algorithms then conduct the same 
computation via our trusted source (scipy.minmize('bfgs')...) and then compute the relative 
error. A trusted textbook will supply the example and also provide an expected answer \citep{Boyd2006}. We will compute the relative error. This is done by taking the norm of the difference of the two vectors over the norm of
the `true' solution. If this is below the threshold we specify: accuracy parameter ($\epsilon_{acc} = 1\%$) 


\item{PSD Test 2 adaptive Step}\

Type: Automatic
					
Initial State: Pending
					
Input/Condition: $\mathbf{A}$ will be set to a matrix specified.
$\mathbf{\vec{b}}$ will be set to a vector specified. The scalar $c$ will be set to 0. We will
use one of each algorithm (DFP, FRCG, and BFGS) with a full step (step size set to 1),
an initial starting choice of the zero vector for the corresponding dimension. For the 
quasi-newton methods we will have an initial Hessian, $H_{0}$ or inverse Hessian, $B_{0}$ set to be the identity 
matrix of the corresponding dimensionality. In this case the adaptive step size requires 
the individual algorithm to calculate the step size at each iteration. This is catered for 
each individual algorithm and adds another degree of complexity.
					
Output/Result: Relative Error of the two vectors (utilizing the norm of a vector). Final result is a Pass/Fail depending on if the norm value is below the err threshold.
					
How test will be performed: We will compute our respective algorithms then conduct the same 
computation via our trusted source a textbook with the specific answer ( also scipy.minmize('bfgs')...) and then compute the relative 
error \citep{Boyd2006}. This is done by taking the norm of the difference of the two vectors over the norm of
the `true' solution. If this is below the threshold we specify: accuracy parameter ($\epsilon_{acc} = 1\%$) 

\item{non-PSD Test 3 adaptive Step}\

Type: Automatic
					
Initial State: Pending
					
Input/Condition: $\mathbf{A}$ will be set to a matrix specified.
$\mathbf{\vec{b}}$ will be set to a vector specified. The scalar $c$ will be set to 0. We will
use one of each algorithm (DFP, FRCG, and BFGS) with a full step (step size set to 1),
an initial starting choice of the zero vector for the corresponding dimension. For the 
quasi-newton methods we will have an initial Hessian, $H_{0}$ or inverse Hessian, $B_{0}$ set to be the identity 
matrix of the corresponding dimensionality. In this case the adaptive step size requires 
the individual algorithm to calculate the step size at each iteration. This is catered for 
each individual algorithm and adds another degree of complexity.
					
Output/Result: Relative Error of the two vectors (utilizing the norm of a vector). Final result is a Pass/Fail depending on if the norm value is below the err threshold.
					
How test will be performed: We will compute our respective algorithms then conduct the same 
computation via our trusted source; a textbook example (also scipy.minmize('bfgs')...) and then compute the relative 
error \citep{Boyd2006}. This is done by taking the norm of the difference of the two vectors over the norm of
the `true' solution. If this is below the threshold we specify: accuracy parameter ($\epsilon_{acc} = 1\%$) 

\item{Test 4 Exact Step}\

Type: Automatic
					
Initial State: Pending
					
Input/Condition: Given the input parameters as specified from a textbook example we will compute the exact step for each algorithm \citep{Boyd2006}. The result will be compared to the result in the textbook; the oracle answer. 
					
Output/Result: Final result is a Pass/Fail depending on if the norm value is below the err threshold.

How test will be performed: Compute the exact step for each algorithm and obtain a value. If the norm between this scalar value and the accepted answer is below the error threshold we make a decision.

\item{Test 5 Gradient}\

Type: Automatic
					
Initial State: Pending
					
Input/Condition: Given a vector and input parameters from a textbook example \citep{Boyd2006}.
					
Output/Result: Final result is a Pass/Fail depending on if the norm value is below the err threshold.

How test will be performed: Compute the gradient for each initial vector and corresponding quadratic form (input parameters define this) and obtain a value. If the norm between this column vector and the accepted answer is below the error threshold we make a decision.


\item{Test 6 Search Direction}\

Type: Automatic
					
Initial State: Pending
					
Input/Condition: Given a vector and input parameters from a textbook example \citep{Boyd2006}.
					
Output/Result: Final result is a Pass/Fail depending on if the norm value is below the err threshold.

How test will be performed: Compute the search for each initial vector and corresponding quadratic form (input parameters define this) and obtain a value. If the norm between this column vector and the accepted answer is below the error threshold we make a decision.

\item{Portability Test 1- Linux\\}

Type: Manual 
					
Initial State: Pending
					
Input: 
					
Output: 
					
How test will be performed: Ensuring we install the dependencies, on a fresh build of ubuntu 22.0
we wish to first install the environment then run the library with the first test for the 
NFR (PSD Test 1 Full step).

\item{Portability Test 2- Windows\\}

Type: Manual 
					
Initial State: Pending
					
Input: 
					
Output: 
					
How test will be performed: Using the conda environment, on windows 11
we wish to first install the environment then run the library with the first test for the 
NFR (PSD Test 1 Full step).

\end{enumerate}

\subsection{Traceability Between Test Cases and Requirements}
Note that TF1 refers to Test Function Requirement 1, and TNF1 refers to Test Non Functional Requirement 1.
\begin{table}[h!]
  \centering
  \begin{tabular}{|c|c|c|c|c|c|c|c|c|c|c|c|c|c|c|c|c|c|c|c|c|c|c|c|}
  \hline        
    & TF1& TF2 & TNF1 & TNF2& TNF3 & TNF4 & TNF5  & TNF6  & TNF7  & TNF8 \\
  \hline
  FR1                    &X &X & & & & & & & & \\ \hline
  FR2        & & X & & & & & & & & \\ \hline
  NFR1      & & &X &X &X &X &X &X & &  \\ \hline
  NFR2 & & X & & & & X & X & X & &  \\ \hline
  NFR3  & & X & & & & X & X & X & &  \\ \hline
  NFR4    & & & & & & & & & X & X \\ \hline
  NFR5     & & & & & & X & X & X & &  \\ \hline
  
  \end{tabular}
  \caption{Traceability Matrix Showing the Connections Between Test Cases and Functional Requirements}
  \label{Table:trace}
  \end{table}

\iffalse 
\section{Unit Test Description}

\wss{This section should not be filled in until after the MIS (detailed design
  document) has been completed.}

\wss{Reference your MIS (detailed design document) and explain your overall
philosophy for test case selection.}  

\wss{To save space and time, it may be an option to provide less detail in this section.  
For the unit tests you can potentially layout your testing strategy here.  That is, you 
can explain how tests will be selected for each module.  For instance, your test building 
approach could be test cases for each access program, including one test for normal behaviour 
and as many tests as needed for edge cases.  Rather than create the details of the input 
and output here, you could point to the unit testing code.  For this to work, you code 
needs to be well-documented, with meaningful names for all of the tests.}

\subsection{Unit Testing Scope}

\wss{What modules are outside of the scope.  If there are modules that are
  developed by someone else, then you would say here if you aren't planning on
  verifying them.  There may also be modules that are part of your software, but
  have a lower priority for verification than others.  If this is the case,
  explain your rationale for the ranking of module importance.}

\subsection{Tests for Functional Requirements}

\wss{Most of the verification will be through automated unit testing.  If
  appropriate specific modules can be verified by a non-testing based
  technique.  That can also be documented in this section.}

\subsubsection{Module 1}

\wss{Include a blurb here to explain why the subsections below cover the module.
  References to the MIS would be good.  You will want tests from a black box
  perspective and from a white box perspective.  Explain to the reader how the
  tests were selected.}

\begin{enumerate}

\item{test-id1\\}

Type: \wss{Functional, Dynamic, Manual, Automatic, Static etc. Most will
  be automatic}
					
Initial State: 
					
Input: 
					
Output: \wss{The expected result for the given inputs}

Test Case Derivation: \wss{Justify the expected value given in the Output field}

How test will be performed: 
					
\item{test-id2\\}

Type: \wss{Functional, Dynamic, Manual, Automatic, Static etc. Most will
  be automatic}
					
Initial State: 
					
Input: 
					
Output: \wss{The expected result for the given inputs}

Test Case Derivation: \wss{Justify the expected value given in the Output field}

How test will be performed: 

\item{...\\}
    
\end{enumerate}

\subsubsection{Module 2}

...

\subsection{Tests for Nonfunctional Requirements}

\wss{If there is a module that needs to be independently assessed for
  performance, those test cases can go here.  In some projects, planning for
  nonfunctional tests of units will not be that relevant.}

\wss{These tests may involve collecting performance data from previously
  mentioned functional tests.}

\subsubsection{Module ?}
		
\begin{enumerate}

\item{test-id1\\}

Type: \wss{Functional, Dynamic, Manual, Automatic, Static etc. Most will
  be automatic}
					
Initial State: 
					
Input/Condition: 
					
Output/Result: 
					
How test will be performed: 
					
\item{test-id2\\}

Type: Functional, Dynamic, Manual, Static etc.
					
Initial State: 
					
Input: 
					
Output: 
					
How test will be performed: 

\end{enumerate}

\subsubsection{Module ?}

...

\subsection{Traceability Between Test Cases and Modules}

\wss{Provide evidence that all of the modules have been considered.}
\fi 
\bibliographystyle{plainnat}

\bibliography{References}

\iffalse 

\newpage
\section{Appendix}

This is where you can place additional information.

\subsection{Symbolic Parameters}

The definition of the test cases will call for SYMBOLIC\_CONSTANTS.
Their values are defined in this section for easy maintenance.

\subsection{Usability Survey Questions?}

\wss{This is a section that would be appropriate for some projects.}

\newpage{}
\section*{Appendix --- Reflection}

The information in this section will be used to evaluate the team members on the
graduate attribute of Lifelong Learning.  Please answer the following questions:

\newpage{}
\section*{Appendix --- Reflection}

\wss{This section is not required for CAS 741}

The information in this section will be used to evaluate the team members on the
graduate attribute of Lifelong Learning.  Please answer the following questions:

\begin{enumerate}
  \item What knowledge and skills will the team collectively need to acquire to
  successfully complete the verification and validation of your project?
  Examples of possible knowledge and skills include dynamic testing knowledge,
  static testing knowledge, specific tool usage etc.  You should look to
  identify at least one item for each team member.
  \item For each of the knowledge areas and skills identified in the previous
  question, what are at least two approaches to acquiring the knowledge or
  mastering the skill?  Of the identified approaches, which will each team
  member pursue, and why did they make this choice?
\end{enumerate}

\fi 

\end{document}